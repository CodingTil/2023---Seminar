\section{Data Gathering Techniques}\label{sec:data_gathering}
\textit{In this section, we will discuss the different approaches to gathering data for visualizing data movements in a program.}

\subsection{Dynamic Analysis}\label{sec:dynamic_analysis}
\textit{Run program and gather data while it is running, using Hardware Counters, Profiling, or Tracing.}

Advantages:
\begin{itemize}
  \item No need for parameterization -> already is compiled for specific hardware
  \item Can be used in combination with actual data -> even more accurate information
\end{itemize}
Disadvantages:
\begin{itemize}
  \item Running a whole program is expensive (time and cost)
  \item Difficult to isolate and analyze specific parts of the program
  \item Very coarse time granularity -> can not measure very short time intervals (hardware counters are not precise enough in aspects of being update / read)
\end{itemize}

\subsection{Static Analysis}\label{sec:static_analysis}
\textit{Analyze the program statically using a compiler}

Advantages:
\begin{itemize}
  \item Can be used to analyze specific parts of the program
  \item Fast and cheap -> No need to run program
\end{itemize}
Disadvantages:
\begin{itemize}
  \item Needs to be parameterized for specific hardware (and often not accurate enough -> might miss some details)
  \item Can not be used in combination with actual data
\end{itemize}

\subsection{Simulation}\label{sec:simulation}
\textit{Simulate the program on a simulator}

Advantages:
\textbf{TODO}
\begin{itemize}
  \item Can be used to analyze specific parts of the program
  \item In between Static and Dynamic Analysis in terms of precision and speed and cost
\end{itemize}
Disadvantages:
\textbf{TODO}
